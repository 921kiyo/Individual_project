\section{Motivation}
\label{sec:motivation}
Reinforcement learning (RL) is a subfield of machine learning concerning where an agent learns how to behave in an environment by interacticing with the environment in order to maximise the total rewards it receives.
The strength of RL is that it can be applied to many different domains where it is unknown to an agent how to perform a task. 
RL has been proven to work well in a number of complex environment, such as dynamic, continuous environments and even a real physical environment, given sufficient training and learning.
Especially together with deep learning (DL), there have been many successful applications of RL in a number of domains,
such as video games \cite{Mnih2015}, the game of Go \cite{Silver2016} and robotics \cite{Levine2015}. 

Despite of these successful applications, however, there are still a number of issues for RL to overcome.
First, most of RL algorithms requires large number of interactions and trials and errors, which is also computationally expense.
Second, most of RL algorithms have no thought process to the decision making, and do not make use of high-level abstract reasoning, 
such as understanding symbolic representations or causal reasoning.
Third, the transfer learning (TL), where the experience that an agent gained to perform one task can be useful in a different task and environment, is limited and the agent performs poorly even on a new but very similar environment or task. 

In order to overcome these limitations, we introduce a new approach by applying Inductive Logic Programming (ILP) into RL senarios. 
ILP is another subfield of machine learning based on logic programming and derives a hypothesis in the form of logic program that, together with background knowledge, entails all of positive examples and none of the negative exmaples. 
% TODO An ILP symtem is a rule-based supervised concept learning over examples, and maps xXXXX
ILP has several advantages compared to RL. 
First, unlike most of statistical machine learning methods, ILP requires a small number of training data due to the presense of language bias which defines what the program can learn.
Second, the learnt program by ILP is expressed with a symbolic representation and therefore is easy to be interpreted by human users.
Third, since the learnt hypothesis is a abstract and general notion, it can be easily applied to a different learning task, thus transfer learning is possible. ILP therefore enables the use of what is previously learnt to similar situations.
The disadvantages of ILP system are that the examples, or training data, have to be clearly defined and, unlike statistical machine learning, cannot be used with ambiguius noisy dataset.
Another disadtantage is that ILP suffers from learning scalability. The search space for a hypothesis defined by the language bias increases with respect to the complexity for learning tasks and slows down learning process.
Despite these shortocomings, there has been a number of progress in ILP, especially in ILP frameworks based on Answer Set Programming (ASP), a declarative logic programming which defines semantics based on Herbrand models (Gelfond and Lifschitz 1988).
Due to the progress on both RL and ILP, we developed an new RL approach by incorporating a new ASP-based ILP framework called Learning from Answer Set (ILASP), which learns a valid move of an environment, 
and uses the learnt hypothesis and background knowledge to generate a sequence of actions in the environment. 

In recent RL research, there is a number of research of introducing symbolic representation into RL in order to achieve more data-efficient learning as well as transparent learning. 
One of the methods is to incorporate symbolic representations into the system \cite{Garnelo2016}. This approach is promising and shows a potential.
However, none of the papers attempt to apply inductive learnig in RL senarios. 
but the combining of ILP and RL has not been explored
Most of the ILP frameworks are also tested in cases where all the environment is known to the agent in advance. 

Since the recent ILP framworks enables to learn a complex hypothesis in a more realistic environments, 
Finally, the recent advance of Inductive Logic Programming (ILP) research has enabled us to apply ILP in more complex situations and there are a number of new algorithms based on Answer Set Programmings (ASPs) that work well in non-monotonic scenarios.
Because of the recent advancement of ILP and RL, it is natual to consider that a combination of both approaches would be the next field to explore.
Therefore my motivation is to combine these two different subfields of machine learnig and devise a new way of RL algorithm in order to overcome some of the RL problems.
% This new approach is tested in a various
% My motivation is to also test the algorithm where the agent does not know the environment in advance, which is the typical case for RL research.
% Second, it resembles how humans reason. Similar to reinforcement learning, there are some aspects of trial-and-error in human learning, but humans exploit reasonings to efficiently learn about their surrounding or situations. 

Particularly since \cite{Garnelo2016}, there have been several researches that further explored the incorporation of symbolic reasoning into RL, but the combining of ILP and RL has not been explored. 

\section{Objectives}
\label{sec:objectives}

The main objective of this project is to provide a proof-of-concept new RL framework using Inductive Logic Programming and to investigate the potentials of improving the learning efficiency and transfer learning capability that current RL algorithms suffer.
This main goal is devided into the following high-level objectives:

% a ASP-based framework called Inductive Learning of Answer Set Programs (ILASP)
% ILASP is a state-of-art ILP method that can be applied to incomplete and more complex environments.
% This initial proof of concept showed promissing preliminary results and as well as limitation of the current framework. 
% Nevertheless, there is avenues for potential improvement, which could be explored in further research.

We did various experiments in grid mazes to highlight property of the learning process and the learning performance is compared with existing reinforcement learning algorithms.
We show that the learning convergence of ILP(RL) is faster than existing RL. 

The objectives of this paper and important aspects of are summarized as follows:

\newcommand\litem[1]{\item{\bfseries #1.\\}}
\begin{enumerate}
\litem{Translation of state transitions into ASP syntax} 
In RL, an agent interacts with an environment by taking an action and observes a state and rewards (MDP). 
Since ASP-based ILP algorithms require their inputs to be specified in ASP syntax, the conversion between MDP and ASP is required.
\litem{Development of learning tasks} ILP frameworks is based on a search space specified by language bias for a learning task, and is needed to be specified by the user. 
Various hyper-parameters for the learning task are also considered.
\litem{Using the learnt concept to execute actions} 
Having learnt a hypothesis using a ILP algorithm, the agent needs to choose an action based on the learnt hypothesis.
We investigate how the agent can effective plan a sequence of actions using the hypothesis.
\litem{Evaluation of the new framework in various environments}
In order to investigate the applicabilities of ILP-based approach, evalauation of the new framework on various senarios is conducted in order to gain
insights of the potentials, especially how it improves the learning process and capability of transfer learning.
\end{enumerate}

% Discrete and deterministic environment. 
This paper is a proof of concept for the new way of reinforcement learning using ILP and therefore there is a limit to extend the current framework can be applied.
More advanced environment such as continuous states or schocastic environment are not considered in this paper. Possibilities of applying these more complex environment are discussed in Section \ref{sec:further_research}.

\section{Contributions}
\label{contributions}
The main contribution of this paper is the development of a novel ILP based approach to reinforcement learning and contributes to an incorporation of ASP-based ILP learning frameworks and Reinforcement Learning by applying a latest ILP framework called Learning from Answer Sets. 
To my knowledge, this is the first attemp that a ILP learning framework is incorporated into a reinforcement learning senario to facilitate learning process.
% and the algorithm learns hypotheses, which is the valid move of the game, which is a very general concept and therefore can be easily applied to a different senarios.

In simple environments, we show that the agent learns rule of the game and reaches an optimal policy faster than existing RL algorithms, learnt concepts is easy to understand for human users.
We also show that the learnt hypothesis is a general concept and can be applied to other environment to mitigate learnig process.

The full hypotheses were learnt in the very early phase of learning and exploration phase. Thus with sufficient exploration, the model of the environment is correct
and therefore it is able to find the optimal policy/path. 

We show that ILP(RL) is able to solve a reduced MDP where the rewards are assumed to be associated with a sequence of actions planned as answer sets.
Although this is a limitated solution, there is a potential to expand it to solve full MDP as discussed in Further Research. 

TODO more details on the strength of the algorithm. 
Validity


\section{Report Outline}
\begin{customthm}{2}
The background of Answer Set Programming, Inductive Logic Programming and Reinforcement Learning necessary for this paper are described.
\end{customthm}

\begin{customthm}{3}
The descriptions of the new framework, called ILP(RL), is explained in details, and we highlight each aspect of learnign steps. 
\end{customthm}

\begin{customthm}{4}
The performance of ILP(RL) is measured in a simple game environment and compared against two existing RL algorithms. In addition, the capability of the transfer learning is experimented.
We also discuss some of the issues we currently face with the architecture.
\end{customthm}

\begin{customthm}{5}
We review previous research on relevant approach. Since there is no research that attemps to apply ILP to RL, we discuss two different groups of reseach: 
RL reseach using answer set programming. The second group is the symbolic representations in RL.  
\end{customthm}

\begin{customthm}{6}
We summarise the framework and experiments of ILP(RL) and discuss the potentials of further research. 
\end{customthm}