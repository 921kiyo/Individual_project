There have been successful applications of deep reinforcement learning (DRL) in a number of domains, such as video games \cite{Mnih2015}, the game of Go \cite{Silver2016} and robotics \cite{Levine2015}. However, there are still a number of issues to overcome with this method.
First, it requires large dataset for training the model, and the learning is very slow and requires significant amount of computation.
Second, it is considered to be a black-box, meaning that the decision making process is unknown to the human user and therefore lacks explanation of the decision making. Third, there is no thought process to the decision making, such as understanding relational representations or planning. To tackle these problems, researchers have explored several different approaches.
One of the methods is to incorporate symbolic representations into the system \cite{Garnelo2016}. This approach is promising and shows a potential.

In this paper, we extend this symbolic representation approach and explore the potential of symbolic machine learning to solve the above issues. There are several advantages of symbolic machine learning. First of all, the decision making mechanism is understandable by humans rather than being black-box.
Second, it resembles how humans reason. Similar to reinforcement learning, there are some aspects of trial-and-error in human learning, but humans exploit reasonings to efficiently learn about their surrounding or situations. They also effectively use previous experience (e.g background knowledge) when encountering similar situations.
Finally, the recent advance of Inductive Logic Programming (ILP) research has enabled us to apply ILP in more complex situations and there are a number of new algorithms based on Answer Set Programmings (ASPs) that work well in non-monotonic scenarios.

Particularly since \cite{Garnelo2016}, there have been several researches that further explored the incorporation of symbolic reasoning into RL, but the combining of ILP and RL has not been explored. Because of the recent advancement of ILP and RL, it is natual to consider that a combination of both approaches would be the next field to explore.

In this paper, our objective is to  explore the incorporation of ILP into RL using Inductive Learning of Answer Set Programs (ILASP), which is a state-of-art ILP method that can be applied to incomplete and more complex environments.

This background report will be part of the final report and is organised as follows: In Chapter \ref{background}, the background of inductive logic programming and reinforcement learning necessary for this paper are described. Chapter \ref{related_work} discusses previous research on relevant approach. Chapter \ref{project_overview} shows the tentative architecture of our new approach, using ILASP to generate a model of the environment. We also discuss some of the issues we currently face with the architecture and plan the implementation. Finally the ethics checklist is provided in Chapter \ref{ethics_checklist}.
